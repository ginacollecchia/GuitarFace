% -----------------------------------------------
% Template for ICMC SMC 2014
% adapted and corrected from the template for SMC 2013,  which was adapted from that of  SMC 2012, which was adapted from that of SMC 2011
% -----------------------------------------------

\documentclass{article}
\usepackage{icmcsmc2014}
\usepackage{times}
\usepackage{ifpdf}
\usepackage[english]{babel}
%\usepackage{cite}

%%%%%%%%%%%%%%%%%%%%%%%% Some useful packages %%%%%%%%%%%%%%%%%%%%%%%%%%%%%%%
%%%%%%%%%%%%%%%%%%%%%%%% See related documentation %%%%%%%%%%%%%%%%%%%%%%%%%%
%\usepackage{amsmath} % popular packages from Am. Math. Soc. Please use the 
%\usepackage{amssymb} % related math environments (split, subequation, cases,
%\usepackage{amsfonts}% multline, etc.)
%\usepackage{bm}      % Bold Math package, defines the command \bf{}
%\usepackage{paralist}% extended list environments
%%subfig.sty is the modern replacement for subfigure.sty. However, subfig.sty 
%%requires and automatically loads caption.sty which overrides class handling 
%%of captions. To prevent this problem, preload caption.sty with caption=false 
%\usepackage[caption=false]{caption}
%\usepackage[font=footnotesize]{subfig}


%user defined variables
\def\papertitle{Detecting the Awesome with GuitarFace}
\def\firstauthor{First author}
\def\secondauthor{Second author}
% \def\thirdauthor{Third author}

% adds the automatic
% Saves a lot of ouptut space in PDF... after conversion with the distiller
% Delete if you cannot get PS fonts working on your system.

% pdf-tex settings: detect automatically if run by latex or pdflatex
\newif\ifpdf
\ifx\pdfoutput\relax
\else
   \ifcase\pdfoutput
      \pdffalse
   \else
      \pdftrue
\fi

\ifpdf % compiling with pdflatex
  \usepackage[pdftex,
    pdftitle={\papertitle},
    pdfauthor={\firstauthor, \secondauthor},
    bookmarksnumbered, % use section numbers with bookmarks
    pdfstartview=XYZ % start with zoom=100% instead of full screen; 
                     % especially useful if working with a big screen :-)
   ]{hyperref}
  %\pdfcompresslevel=9

  \usepackage[pdftex]{graphicx}
  % declare the path(s) where your graphic files are and their extensions so 
  %you won't have to specify these with every instance of \includegraphics
  \graphicspath{{./figures/}}
  \DeclareGraphicsExtensions{.pdf,.jpeg,.png}

  \usepackage[figure,table]{hypcap}

\else % compiling with latex
  \usepackage[dvips,
    bookmarksnumbered, % use section numbers with bookmarks
    pdfstartview=XYZ % start with zoom=100% instead of full screen
  ]{hyperref}  % hyperrefs are active in the pdf file after conversion

  \usepackage[dvips]{epsfig,graphicx}
  % declare the path(s) where your graphic files are and their extensions so 
  %you won't have to specify these with every instance of \includegraphics
  \graphicspath{{./figures/}}
  \DeclareGraphicsExtensions{.eps}

  \usepackage[figure,table]{hypcap}
\fi

%setup the hyperref package - make the links black without a surrounding frame
\hypersetup{
    colorlinks,%
    citecolor=black,%
    filecolor=black,%
    linkcolor=black,%
    urlcolor=black
}


% Title.
% ------
\title{\papertitle}

% Authors
% Please note that submissions are NOT anonymous, therefore 
% authors' names have to be VISIBLE in your manuscript. 
%
% Single address
% To use with only one author or several with the same address
% ---------------
%\oneauthor
%   {\firstauthor} {Affiliation1 \\ %
%     {\tt \href{mailto:author1@smcnetwork.org}{author1@smcnetwork.org}}}

%Two addresses
%--------------
% \twoauthors
%   {\firstauthor} {Affiliation1 \\ %
%     {\tt \href{mailto:author1@smcnetwork.org}{author1@smcnetwork.org}}}
%   {\secondauthor} {Affiliation2 \\ %
%     {\tt \href{mailto:author2@smcnetwork.org}{author2@smcnetwork.org}}}

% Three addresses
% --------------
 \twoauthors
   {\firstauthor} {Regina Collecchia \\ %
     {\tt \href{mailto:colleccr@ccrma.stanford.edu}{colleccr@ccrma.stanford.edu}}}
   {\secondauthor} {Roshan Vidyashankar \\ %
     {\tt \href{mailto:rvid@ccrma.stanford.edu}{rvid@ccrma.stanford.edu}}}
%   {\thirdauthor} { Affiliation3 \\ %
%     {\tt \href{mailto:author3@smcnetwork.org}{author3@smcnetwork.org}}}


% ***************************************** the document starts here ***************
\begin{document}
%
\capstartfalse
\maketitle
\capstarttrue
%
\begin{abstract}
It is difficult to tackle the problem of identifying ``awesomeness" without facing serious judgment calls on musical taste. One thing that does give away the joy and energy contained in a moment is the facial expression of a performer. Aiming to recognize the face a guitarist makes when he or she is doing something awesome, we created \textit{GuitarFace}, a Mac application with facial recognition via the OpenCV library. The user is a musician practicing their instrument either alone or across the network with a friend. Gameplay is a mix of MIDI guitar input visualization, counts of musical events like big jumps in pitch or power chords, and events initiated by making a guitar face. In another sense, GuitarFace is the beginnings of a ``funny face" detector.
\end{abstract}
%

\section{Introduction}\label{sec:introduction}

\section{Facial Recognition Techniques}\label{sec:frt}
Using the built in face detector in OpenCV, the rectangular area around the face was detected with high accuracy and only a few lines of code. The strongest feature that we assimilated to ``guitar face" was an open mouth. We tried a few different techniques to detect this feature with interesting results and mixed success.
\subsection{Color comparison}
 We assumed that the face was straight on. 
\subsection{Size comparison}

\subsection{Machine learning}

\section{Implementation}
\subsection{Features}

\section{Evaluation}\label{sec:eval}

\section{Conclusion and Future Work}\label{conclusion}

%\subsection{Equations}
%Equations should be placed on separated lines and numbered.
%The number should be on the right side, in parentheses.
%\begin{equation}
%E=mc^{2}.
%\label{eq:Emc2}
%\end{equation}


\subsection{Figures, Tables and Captions}

%%%%%%%%%%%%%%%%%%%%%%%%%%%%%%%%%%%%%%%%%%%%%%%%%%%%%%%%%%%%%%%%%%%%%%%%%%%%%
%bibliography here
\bibliography{smacsmc2014template}
% http://opencv.org
\end{document}
